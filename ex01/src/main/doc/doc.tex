\documentclass[a4paper,headings=small]{scrartcl}
\KOMAoptions{DIV=12}

\usepackage[utf8x]{inputenc}
\usepackage{amsmath}
\usepackage{graphicx}
\usepackage{multirow}
\usepackage{listings}

% define style of numbering
\numberwithin{equation}{section} % use separate numbering per section
\numberwithin{figure}{section}   % use separate numbering per section

% instead of using indents to denote a new paragraph, we add space before it
\setlength{\parindent}{0pt}
\setlength{\parskip}{10pt plus 1pt minus 1pt}

\title{Automatic Image Analysis - WS12/13 \\ Excercise 1 \\ \emph{Load, Modify and Save Images with OpenCV}}
\author{Team e: Marcus Grum, Robin Vobruba, Robert Koppisch, Nicolas Brieger}
\date{\today}

\pdfinfo{%
  /Title    (Automatic Image Analysis - WS12/13 - Excercise 1 - Load, Modify and Save Images with OpenCV)
  /Author   (Team e: Marcus Grum, Robin Vobruba, Robert Koppisch, Nicolas Brieger)
  /Creator  ()
  /Producer ()
  /Subject  ()
  /Keywords ()
}

% Simple picture reference
%   Usage: \image{#1}{#2}{#3}
%     #1: file-name of the image
%     #2: percentual width (decimal)
%     #3: caption/description
%
%   Example:
%     \image{myPicture}{0.8}{My huge house}
%     See fig. \ref{fig:myPicture}.
\newcommand{\image}[3]{
	\begin{figure}[htbp]
		\centering
		\includegraphics[width=#2\textwidth]{#1}
		\caption{#3}
		\label{fig:#1}
	\end{figure}
}


\begin{document}


\maketitle



\section{Load, Modify and Save Image}


\subsection{Load}

You can see the plot of the original \emph{colored} picture as source
in fig. \ref{fig:../../../target/input}.

\image{../../../target/input}{0.5}{%
		Original picture (\emph{colored}).}



\subsection{Modify}


When a photograph is taken, images are captured with a limited dyanmic range because of 
the limitations of camera sensors. In using a histogram equalization, 
we use a modification that shall impress our tutor:
The brightness distribution of an image can be equalized and 
thereby we increase the contrast of the image and in a sense additionally the dynamic range.
For doing so, the following function will be used:
\begin{lstlisting}
cvEqualizeHist ( img1, out );
\end{lstlisting}
For this, the \emph{colored} picture given as source can't be used,
because of the requirements of the function. It has to be prepared a \emph{grey} version of it,
which we call \emph{img1}. The function creates an \emph{optimized} image, which we call \emph{out}.
This variable will be returned.
For the comparison, all three picture versions will be saved: 
the \emph{colored} image, 
the \emph{grey} image and
the \emph{optimized} image.


\subsection{Save}

You can see the plot of the optimized \emph{grey} picture as it is saved 
in fig. \ref{fig:../../../target/result_grey_optimized}.

\image{../../../target/result_grey_optimized}{0.5}{%
		Modified picture (\emph{grey}) and (\emph{optimized}).}

\newpage

In comparison with the original \emph{grey} picture in fig. \ref{fig:../../../target/result_grey}., you can see the optimization quite well and the application of the Histogram Equalization was a great success! 

\image{../../../target/result_grey}{0.5}{%
		Modified picture (\emph{grey}) as original.}

\end{document}
