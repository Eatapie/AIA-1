\documentclass[a4paper,headings=small]{scrartcl}
\KOMAoptions{DIV=12}

\usepackage[utf8x]{inputenc}
\usepackage{amsmath}
\usepackage{graphicx}
\usepackage{multirow}
\usepackage{listings}

% define style of numbering
\numberwithin{equation}{section} % use separate numbering per section
\numberwithin{figure}{section}   % use separate numbering per section

% instead of using indents to denote a new paragraph, we add space before it
\setlength{\parindent}{0pt}
\setlength{\parskip}{10pt plus 1pt minus 1pt}

\title{Automatic Image Analysis - WS12/13 \\ Excercise 1 \\ \emph{Load, Modify and Save Images with OpenCV}}
\author{Team e: Marcus Grum, Robin Vobruba, Robert Koppisch, Nicolas Brieger}
\date{\today}

\pdfinfo{%
  /Title    (Automatic Image Analysis - WS12/13 - Excercise 1 - Load, Modify and Save Images with OpenCV)
  /Author   (Team e: Marcus Grum, Robin Vobruba, Robert Koppisch, Nicolas Brieger)
  /Creator  ()
  /Producer ()
  /Subject  ()
  /Keywords ()
}

% Simple picture reference
%   Usage: \image{#1}{#2}{#3}
%     #1: file-name of the image
%     #2: percentual width (decimal)
%     #3: caption/description
%
%   Example:
%     \image{myPicture}{0.8}{My huge house}
%     See fig. \ref{fig:myPicture}.
\newcommand{\image}[3]{
	\begin{figure}[htbp]
		\centering
		\includegraphics[width=#2\textwidth]{#1}
		\caption{#3}
		\label{fig:#1}
	\end{figure}
}


\begin{document}


\maketitle



\section{Load, Modify and Save Image}


\subsection{Load}

You can see the plot of the original \emph{colored} picture as source
in fig. \ref{fig:../../../target/input}.

\image{../../../target/input}{0.5}{%
		Original picture (\emph{colored}).}



\subsection{Modify}

The modification that shall impress our tutor will take the emph{colored} picture as source 
and create a \emph{grey} version of it. Herefor, the following command uses the \emph{colored} 
picture \emph{img} and creates the \emph{grey} image \emph{gray\_image}:
\begin{lstlisting}
cvtColor( img, gray_image, CV_RGB2GRAY ).
\end{lstlisting}

\newpage

\subsection{Save}

You can see the plot of the modified \emph{grey} picture as saved outcome
in fig. \ref{fig:../../../target/result}.

\image{../../../target/result}{0.5}{%
		Modified picture (\emph{grey}).}



\end{document}
